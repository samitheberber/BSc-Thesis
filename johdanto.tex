\section{Johdanto}

Perinteisesti ohjelmointia on opetettu kirjallista materiaalia esittelemällä ja
muutamalla materiaaliin pohjautuvalla yksinkertaisella ohjelmointitehtävällä.
Aloittelijoille suunnattu oppimateriaali käsittelee yleisten ohjelmointiin
liittyvien rakenteiden sijaan ohjelmointikielen rakenteita
\cite{Caspersen:2006:NPO:1176617.1176741, Vihavainen:2011:EAM:1953163.1953196}.
Opiskelijoille esitetään ongelma ja sen ratkaiseva valmis ohjelma. Ohjelman
rakenteet käydään lävitse materiaalissa tai luennolla.

Perinteisellä tavalla järjestetyllä kurssilla opiskelijat oppivat lähinnä
ratkaisemaan tietyntyyppisiä tehtäviä eivätkä yleistä ongelmanratkaisua
\cite{Gries:1974:WTI:953057.810447}. Valmiita ohjelmia esiteltäessä
opiskelijoille syntyy kuva, että ohjelmat syntyvät yhdellä askeleella sopivia
rakenteita yhdistelemällä; sen sijaan ohjelman todellinen kehitysprosessi ilmene
opiskelijalle.  Seurauksena tästä on se, että opiskelijat ymmärtävät eri
rakenteet, mutta eivät osaa luoda toimivaa ohjelmaa niitä yhdistelemällä.

Ohjelmointiprosessin jäädessä piiloon, opiskelijat eivät pysty ratkaisemaan
annettuja haasteita tai heille kehittyy oma prosessi
\cite{Caspersen:2006:NPO:1176617.1176741}. Yleensä itsekehittynyt prosessi
johtaa epäpäteviin ja huonosti suunniteltuihin lopputuloksiin. Tämä johtuu
siitä, että yleensä opiskelija valitsee ensimmäisen mieleen tulleen ratkaisun.
Moni on tyytyväinen löydettyään edes yhden ratkaisun eivätkä ole halukkaita
tutustumaan muihin ratkaisuvaihtoehtoihin.

Kursseilla käytetyt esimerkkiohjelmat ovat yleensä pieniä ja koostuvat
yksinkertaisista askelista, mikä ei vastaa todellisuutta. Lisäksi ne opettavat
vain kielen rakenteita \cite{Astrachan:1995:ACA:199691.199694}. Merkittävän
kokoiset tehtävät vaativat ulkoista tukea \cite{Kolling:2008} eli sopivaa
kehitysympäristöä (IDE). Tavallisesti kehitysympäristö ei anna hyvää tukea
opettamis- ja oppimisprosessille, koska se on liian monimutkainen ja
ohjelmointikielen tuki on vajaavainen.

Asiat ovat näiltä osin menneet paljon parempaan suuntaan viime vuosien aikana.
Joka tapauksessa kurssin alussa ohjelmointiprosessin sijaan täytyy opetella
kehitysympäristön käyttö, mikä luonnollisesti vie aikaa ohjelmoinnin
opetukselta. Kehitysympäristön käyttö kuitenkin vähentää kieleen liittyvien
rakenteiden opettelua, mutta silloin on entistä tärkeämpää ymmärtää, miksi
kehitysympäristön tuottama koodi on tietynlaista. Jotta opiskelijoilla on
mahdollisuus ymmärtää kirjoittamansa ohjelmakoodi ja miksi se toimii kuten
toimii, on tärkeää ettei alussa käytetä kaikkia IDEn ominaisuuksia vaan
opiskelijat luovat itse kaiken alusta asti.

Javan ottaminen opetuskieleksi on tuonut hyviä ja huonoja asioita mukanaan
\cite{Kolling:2008}. Sille on kehitetty hyviä kehitysympäristöjä ja se on
suhteellisen vakaa kieli. Kuitenkin sen käyttöönotto on lisännyt enemmän
keskittymistä kielen rakenteisiin. Yksinkertainen Java ohjelma vaatii
kohtuullisen paljon Javan omaa syntaksia. Täten kielen syntaksia on osattava
kohtalainen määrä ennen kuin yleisiä ohjelmointirakenteita voi opettaa.

Ohjelmoinnin oppiminen ei saa olla ainoana tavoitteena vaan myös koodin laadun
kehitys on tärkeää. Jos ohjelmointikursseilla opetetaan järjestelmällisyyteen
pohjautuvaa ongelmanratkaisuprosessia (käsitellään luvussa
\ref{järjestelmällisyyteen pohjautuva ongelmanratkaisuprosessi}), niin se ei
ainoastaan auta opiskelijoita kehittämään ratkaisuja vaan myös parantaa
ratkaisujen laatua.

Tässä työssä selvitetään miten ohjelmointiprosessia voidaan tuoda esille
ohjelmoinnin opetuksessa. Luvussa 2 tarkastellaan ongelmanratkaisuprosessia ja
sen soveltamista ohjelmointiin. Luvussa 3 tarkastellaan kisällioppimista ja
miten se tukee ohjelmointiprosessin tuontia esille.
