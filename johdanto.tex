\section{Johdanto}

Perinteisesti ohjelmointia on opetettu kirjallista materiaalia näyttämällä ja
muutamalla materiaaliin pohjautuvalla yksinkertaisella ohjelmointitehtävällä.
Tällä tavalla järjestetyllä kurssilla opiskelijat oppivat lähinnä ratkaisemaan
tietyntyyppisiä tehtäviä eikä yleistä ongelmanratkaisua
\cite{Gries:1974:WTI:953057.810447}.

Aloittelijoille suunnatut tekstit käsittelevät yleisten ohjelmointiin liittyvien
rakenteiden sijaan ohjelmointikielen rakenteisiin
\cite{Caspersen:2006:NPO:1176617.1176741, Vihavainen:2011:EAM:1953163.1953196}.
Opiskelijoille esitetään ongelma ja sen ratkaiseva valmis ohjelma. Ohjelman
rakenteet käydään lävitse materiaalissa tai luennolla. Tällöin opiskelijoille
syntyy kuva, että ohjelmat syntyvät yhdellä askeleella sopivia rakenteita
yhdistelemällä ja ohjelman todellinen kehitysprosessi ei ilmene opiskelijalle.
Seurauksena tästä on se, että opiskelijat ymmärtävät eri rakenteet, mutta eivät
osaa luoda niitä yhdistelemällä toimivaa ohjelmaa.

Esimerkkiohjelmat ovat yleensä pieniä ja koostuvat yksinkertaisista askelista,
eikä tämä vastaa todellisuutta. Lisäksi ne opettavat vain kielen rakenteita
\cite{Astrachan:1995:ACA:199691.199694}. Merkittävän kokoiset tehtävät vaativat
ulkoista tukea \cite{Kolling:2008}. Ulkoisena tukena tarkoitan kehitysympäristöä
(IDE). Tavallisesti tällainen ympäristö ei anna hyvää tukea opettamis- ja
oppimisprosessille. Javan ottaminen opetuskieleksi toi hyviä ja huonoja asioita
mukanaan. Sille on hyviä kehitysympäristöjä ja se on suhteellisen vakaa kieli.
Kuitenkin se lisäsi enemmän keskittymistä kielen rakenteisiin.

Ohjelmointiprosessin olessa salassa, opiskelijat eivät pysty ratkaisemaan
annettuja haasteita tai heille kehittyy oma prosessi
\cite{Caspersen:2006:NPO:1176617.1176741}. Yleensä itsekehittynyt prosessi
johtaa huonoihin lopputuloksiin. Ohjelmoinnin oppiminen ei saa olla ainoana
tavoitteena vaan myös koodin laadun kehitys on tärkeää.

Olemassa olevan koodin lukeminen ja muokkaaminen ovat yleisiä tehtäviä kelle
tahansa ohjelmoijalle \cite{Kolling:2008}. Oppimisen kannalta on suotavaa käydä
lävitse muiden kirjottamaa koodia varhaisessa vaiheessa. Muiden koodin
korjaaminen auttaa hahmottamaan, mitkä ovat hyviä käytäntöjä ja mitä hyötyä on
ylläpidettävästä koodista.
