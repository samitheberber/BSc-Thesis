\section{Johdanto}

- Mikä on ACID?

- Mikä on SQL?

- Mikä on relaatiotietokanta?

- Relaatiotietokannan rajoituksia

* Skaalautuvuus\\
o Tietokantaa ajetaan tehokkaammalla koneella\\
o Hajautettu skaalaus ei toimi helposti relaatiokantojen kanssa.

* Monimutkaisuus\\
o Data on muutettava sopimaan tauluihin\\
o Tietosisältö voi muuttua monimutkaiseksi, vaikeaksi ja hitaaksi käsitellä.

* SQL-kieli\\
o Sopii hyvin rakenteellisen tiedon kanssa, muttei muun tyyppisen tiedon kanssa.\\
o Voi johtaan suureen määrään monimutkaista koodia ja ei sovi ketterään kehitykseen.

* Suuri joukko ominaisuuksia\\
o Käyttäjät eivät usein tarvitse kaikkia ominaisuuksia, kuten ei myöskään niiden tuomaa kustannusta ja monimutkaisuutta.

- Mikä on NoSQL?

- Millaisia toteutustapoja on?

* Avain-arvo -parit\\
* sarakepainotteiset tietokannat (column-oriented databases)

- Mitä etua NoSQL:stä on SQL:ään verrattuna?

* Nopea tiedonkäsittely\\
o ACIDista huolehtiminen tekee relaatiokannoista hitaampia\\
o Tietomallit yleensä yksinkertaisempia

- Millaisia huolia NoSQL:stä on?

* Kyselyjen kirjoittaminen\\
* Ei tue natiivisti ACIDia\\
* Yhdenmukaisuus\\
* NoSQL ei ole vielä tuttu kaikille\\
* Asiakastuki ja hallintatyökalut puuttuvat
