\section{Johdanto}

Perinteisesti ohjelmointia on opetettu kirjallista materiaalia esittelemällä ja
muutamalla materiaaliin pohjautuvalla yksinkertaisella ohjelmointitehtävällä.
Tällä tavalla järjestetyllä kurssilla opiskelijat oppivat lähinnä ratkaisemaan
tietyntyyppisiä tehtäviä eivät yleistä ongelmanratkaisua
\cite{Gries:1974:WTI:953057.810447}.

Aloittelijoille suunnattu oppimateriaali käsittelee yleisten ohjelmointiin
liittyvien rakenteiden sijaan ohjelmointikielen rakenteita
\cite{Caspersen:2006:NPO:1176617.1176741, Vihavainen:2011:EAM:1953163.1953196}.
Opiskelijoille esitetään ongelma ja sen ratkaiseva valmis ohjelma. Ohjelman
rakenteet käydään lävitse materiaalissa tai luennolla. Tällöin opiskelijoille
syntyy kuva, että ohjelmat syntyvät yhdellä askeleella sopivia rakenteita
yhdistelemällä, eikä ohjelman todellinen kehitysprosessi ilmene opiskelijalle.
Seurauksena tästä on se, että opiskelijat ymmärtävät eri rakenteet, mutta eivät
osaa luoda toimivaa ohjelmaa niitä yhdistelemällä.

Kursseilla käytetyt esimerkkiohjelmat ovat yleensä pieniä ja koostuvat
yksinkertaisista askelista, mikä ei vastaa todellisuutta. Lisäksi ne opettavat
vain kielen rakenteita \cite{Astrachan:1995:ACA:199691.199694}. Merkittävän
kokoiset tehtävät vaativat ulkoista tukea \cite{Kolling:2008} eli sopivaa
kehitysympäristöä (IDE). Tavallisesti tällainen ympäristö ei anna hyvää tukea
opettamis- ja oppimisprosessille. %TODO: Miksei?

Javan ottaminen opetuskieleksi on tuonut hyviä ja huonoja asioita mukanaan.
Sille on kehitetty hyviä kehitysympäristöjä ja se on suhteellisen vakaa kieli.
Kuitenkin sen käyttöönotto lisäsi enemmän keskittymistä kielen rakenteisiin.

Ohjelmointiprosessin jäädessä piiloon, opiskelijat eivät pysty ratkaisemaan
annettuja haasteita tai heille kehittyy oma prosessi
\cite{Caspersen:2006:NPO:1176617.1176741}. Yleensä itsekehittynyt prosessi
johtaa huonoihin lopputuloksiin. %TODO: Millaisiin? Seurauksia? Esimerkkejä!

Ohjelmoinnin oppiminen ei saa olla ainoana tavoitteena vaan myös koodin laadun
kehitys on tärkeää.

Olemassa olevan koodin lukeminen ja muokkaaminen ovat tavallisimpia tehtäviä
kelle tahansa ohjelmoijalle \cite{Kolling:2008}. Oppimisen kannalta on suotavaa
käydä lävitse muiden kirjottamaa koodia varhaisessa vaiheessa. Muiden koodin
korjaaminen auttaa hahmottamaan, mitkä ovat hyviä käytäntöjä ja mitä hyötyä on
siitä, että koodin ylläpidettävyys on hyvä.
