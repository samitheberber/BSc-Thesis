\section{Yhteenveto}

Ohjelmoinnin opetuksessa on aiemmin keskitytty kielen rakenteisiin
ohjelmarakenteiden ja ongelmanratkaisuprosessin sijaan. Kisällioppiminen tuo
tähän parannusta, mutta ei yksinään ratkaise kaikkea. Oikeanlaisilla tehtävillä
ja ohjauksella pajatilanteesta saa paljon irti.

Hyvä koodi on tärkeä osa ohjelmointia. Se sisältää piirteitä eri ohjelmointiin
liittyvistä ongelmanratkaisuprosesseista. Tämä on toisiaan ruokkiva ympyrä,
josta on syytä pitää kiinni.

Luentojen rooli on edelleen todella tärkeä. Luennoilla ei riitä, että esitellään
ohjelmointirakenteet ja valmis ratkaisu. On tärkeää tuoda ilmi koko prosessi
ongelman analysoinnista oikeiden rakenteiden valitsemiseen ja valmiin ohjelman
saamiseen usean refaktorointikierroksen jälkeen valmiiksi.

Videot ovat tärkeä lisä materiaaliin. Niihin voi palata tehtäviä tehdessä ja
niissä saa korostettua ja näytettyä asioita, joihin kirjallinen materiaali ei
kykene.

Kaksisuuntainen palaute on tärkeää opintojen suunnittelussa. Opiskelija saa
palautetta edistymisestään ja ohjaaja saa tiedon asioiden sisäistämisestä.
Luentoja suunnitellessa tämä tieto on tärkeä, jotta voidaan keskittyä epäselviin
asioihin.
