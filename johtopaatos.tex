\section{Johtopäätös}

Ohjelmoinnin opetuksessa on keskitytty kielenrakenteisiin ohjelmarakenteiden ja
ongelmaratkaisuprosessin sijaan. Kisällioppiminen tuo tähän parannusta, mutta ei
yksinään ratkaise kaikkea. Oikeanlaisilla tehtävillä ja ohjauksella
pajatilanteesta saa täyden tehon irti.

Luentojen rooli on edelleen todella tärkeä. Luennoilla ei riitä, että esitellään
ohjelmointirakenteet ja valmis ratkaisu. On tärkeää tuoda ilmi koko prosessi
ongelman analysoinnista oikeiden rakenteiden valitsemiseen ja valmiin ohjelman
saamiseen usean refaktorointikierroksen jälkeen valmiiksi.

Luennot on tärkeä videoida, jotta niihin voi palata jälkikäteen ja kerrata
luennon asiat, kunnes ne ovat selviä. Eri ihmisillä kestää eri aika sisäistää
asia ja omaan tahtiin kertaaminen on tällöin eduksi.

Kaksisuuntainen palaute on tärkeää opintojen suunnittelussa. Opiskelija saa
palautetta edistymisestään ja ohjaaja saa tiedon asioiden sisäistämisestä.
Luentoja suunnitellessa tämä tieto on tärkeä, jotta voidaan keskittyä epäselviin
asioihin.
