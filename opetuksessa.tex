\section{Opetukseen sisällyttäminen}

On erityisen tärkeää tuoda ongelmanratkaisuprosessi esille opetuksessa.
Oppikirjoista ei voi opetella tätä vaan tarvitaan vaihtoehtoisia
oppimateriaaleja, joita käsitellään tässä luvussa. Koko luku perustuu lähteeseen
\cite{Bennedsen:2008}.

Liitutaulun käyttö on yleistä matematiikan opetuksessa ja sitä voidaan hyödyntää
samoin ohjelmoinnin opetuksessa. Opiskelijat ovat mukana kehityksessä, joten he
voivat kommentoida ja ohjata toteutusta. Ongelmana liitutaulun käytössä on se,
että tila on todella rajallista, joten suurempia ohjelmia sillä ei voi esittää.

Piirtoheittimen käyttö ratkaisee tilan puuteen, mutta valmiiksi tehtyihin
kalvoihin opiskelijat eivät pääse vaikuttamaan. Ratkaisun esitystahti on
haastava pitää sopivana, jotta opiskelijat ymmärtävät varmasti, mitä missäkin
tehtiin ja miksi. Videotykiltä esitetyissä esimerkeissä on täsmälleen sama
ongelma, joten uusi teknologia ei ratkaise ongelmaa tällä tavoin.

Teknologia mahdollistaa tarvittavat olosuhteet. Liitutaulun käytössä opiskelijat
otetaan mukaan ja piirtoheitinkalvoihin saa tarpeeksi isoja kokonaisuuksia.
Tietokone ja videotykki mahdollistavat kehitystyökalujen käytön luennolla,
jolloin opiskelijat näkevät oikeanlaista ohjelmointia ja voivat itse vaikuttaa
lopputulokseen. Kuitenkin luento kestää vain määrätyn ajan, joten kovin
monimutkaisia ohjelmia ei voida esittää. Lisäksi opiskelija näkee
ohjelmointiprosessin vain kerran eikä hän voi palata siihen kuten esimerkiksi
kirjan esimerkkiin, jonka voi lukea monta kertaa.

Onneksi nykyään luennot on mahdollista nauhoittaa, jolloin opiskelijat voivat
katsoa luentovideon moneen kertaan, kunnes ymmärtävät miten ohjelmointiprosessi
toimi luennolla esitetyssä ongelmassa. Tämän jälkeen heidän on helppo palata
tarkistamaan kehitysprosessin eteneminen, kun tehtäviä tehdessä tarvitsevat
apua. Eri ihmisillä kestää eri aika sisäistää luennon asia, joten
kertausmahdollisuus on eduksi.

On tärkeää, että videoissa on sisällysluettelo, jolloin opiskelijat löytävät
helposti tarvitsemansa kohdan videosta. Mitä helpommin opiskelija saa videoilta
etsimänsä tiedon, sitä vähemmän kirjallisesta materiaalista on tarve etsiä
tiettyä asiaa.

Liian käsikirjoitetut luennot ovat huonoja, sillä silloin tilanne ei vastaa
todellisuutta ja opiskelijat voivat tylsistyä. Sopiva määrä improvisointia ja
mahdolliset virheet ovat hyväksi, koska silloin nähdään, että ohjelmoidessa on
luontaista tehdä välillä virheitä. Myös kääntäjän antamia virheilmoituksia on
syytä käsitellä, jotta ymmärretään, mistä virheet johtuvat.

Luennoilla on tärkeä näyttää myös refaktorointia. Aluksi tuetettu ratkaisu tekee
asiansa, mutta se ei välttämättä ole hyvää koodia. Refaktorointi on yleinen osa
ohjelmointia, joten se tulee tuoda siten myös esille ja näyttää miten
refaktoroidaan.

Olemassa olevan koodin lukeminen ja muokkaaminen ovat tavallisimpia tehtäviä
kelle tahansa ohjelmoijalle \cite{Kolling:2008}. Oppimisen kannalta on suotavaa
käydä lävitse muiden kirjottamaa koodia varhaisessa vaiheessa. Muiden koodin
korjaaminen auttaa hahmottamaan, mitkä ovat hyviä käytäntöjä ja mitä hyötyä on
siitä, että koodin ylläpidettävyys on hyvä.
