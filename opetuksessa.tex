\section{Prosessin tuominen opetukseen}

On erityisen tärkeää tuoda ongelmanratkaisuprosessi esille opetuksessa.
Oppikirjoista ei voi opetella tätä vaan tarvitaan vaihtoehtoisia
oppimateriaaleja, joita käsitellään tässä luvussa.

\subsection{Vaihtoehtoiset esitystavat}

Koko luku perustuu lähteeseen \cite{Bennedsen:2008}.

Liitutaulun käyttö on yleistä matematiikan opetuksessa ja sitä voidaan hyödyntää
samoin ohjelmoinnin opetuksessa. Opiskelijat ovat mukana kehityksessä, joten he
voivat kommentoida ja ohjata toteutusta. Ongelmana liitutaulun käytössä on se,
että tila on todella rajallista, joten suurempia ohjelmia sillä ei voi esittää.

Piirtoheittimen käyttö ratkaisee tilan puuteen, mutta valmiiksi tehtyihin
kalvoihin opiskelijat eivät pääse vaikuttamaan. Ratkaisun esitystahti on
haastava pitää sopivana, jotta opiskelijat ymmärtävät varmasti, mitä missäkin
tehtiin ja miksi. Videotykiltä esitetyissä esimerkeissä on täsmälleen sama
ongelma, joten uusi teknologia ei ratkaise ongelmaa tällä tavoin.

Teknologia mahdollistaa tarvittavat olosuhteet. Liitutaulun käytössä opiskelijat
otetaan mukaan ja piirtoheitinkalvoihin saa tarpeeksi isoja kokonaisuuksia.
Tietokone ja videotykki mahdollistavat kehitystyökalujen käytön luennolla,
jolloin opiskelijat näkevät oikeanlaista ohjelmointia ja voivat itse vaikuttaa
lopputulokseen. Kuitenkin luento kestää vain määrätyn ajan, joten kovin
monimutkaisia ohjelmia ei voida esittää. Lisäksi opiskelija näkee
ohjelmointiprosessin vain kerran eikä hän voi palata siihen kuten esimerkiksi
kirjan esimerkkiin, jonka voi lukea monta kertaa.

Onneksi nykyään luennot on mahdollista nauhoittaa, jolloin opiskelijat voivat
katsoa luentovideon moneen kertaan, kunnes ymmärtävät miten ohjelmointiprosessi
toimi luennolla esitetyssä ongelmassa. Tämän jälkeen heidän on helppo palata
tarkistamaan kehitysprosessin eteneminen, kun tehtäviä tehdessä tarvitsevat
apua. Eri ihmisillä kestää eri aika sisäistää luennon asia, joten
kertausmahdollisuus on eduksi.

On tärkeää, että videoissa on sisällysluettelo, jolloin opiskelijat löytävät
helposti tarvitsemansa kohdan videosta. Mitä helpommin opiskelija saa videoilta
etsimänsä tiedon, sitä vähemmän kirjallisesta materiaalista on tarve etsiä
tiettyä asiaa.

Liian käsikirjoitetut luennot ovat huonoja, sillä silloin tilanne ei vastaa
todellisuutta ja opiskelijat voivat tylsistyä. Sopiva määrä improvisointia ja
mahdolliset virheet ovat hyväksi, koska silloin nähdään, että ohjelmoidessa on
luontaista tehdä välillä virheitä. Myös kääntäjän antamia virheilmoituksia on
syytä käsitellä, jotta ymmärretään, mistä virheet johtuvat.

\subsection{Hyvä koodi kunniaan}

Luennoilla ja materiaalissa on tärkeä tähdentää, että oma koodi on vasta sitten
valmis, kun se on hyvää koodia. Luennoilla on tärkeä näyttää myös
refaktorointia. Aluksi tuetettu ratkaisu tekee asiansa, mutta se ei välttämättä
ole hyvää koodia. Refaktorointi on yleinen osa ohjelmointia, joten se tulee
tuoda siten myös esille ja näyttää miten refaktoroidaan.

Jotta voidaan refaktoroida, tarvitaan testejä. Testit eivät tarkoita välttämättä
yksikkötestejä vaan pääohjelmaa, jossa määritellään, mitä ohjelma tekee.
Ohjelman toimivuuden näkee sitten pääohjelman ajamalla ja tulosteet
tarkistamalla. Tämä on varsin kevyt tapa toteuttaa yksinkertaiset testit.
Myöhemmin voidaan mainita, että oikeasti tulee käyttää yksikkötestejä, kuten
palautuksen yhteydessä on käytetty. Opiskelijoiden ei tarvitse itse osata
ensimmäisten ohjelmointikurssien jälkeen kirjoittaa yksikkötestejä, mutta
yleinen testaamisen ymmärtäminen on syytä osata.

Helsingin yliopistolla ohjelmointikurssien jälkimmäisellä puoliskolla
opiskelijat osallistuvat ohjelmistotekniikan menetelmät -kurssille, jossa
käsitellään myös yksikkötestausta. Tällä kurssilla tulee olla yhtenä osana
yksikkötestien lukeminen ja pienessä määrin kirjoittaminen. Tämä luo pohjatiedot
harjoitustyökursseille, joissa tulee omalle ohjelmalle tehdä yksikkötestit.

Olemassa olevan koodin lukeminen ja muokkaaminen ovat tavallisimpia tehtäviä
kelle tahansa ohjelmoijalle \cite{Kolling:2008}. Oppimisen kannalta on suotavaa
käydä lävitse muiden kirjottamaa koodia varhaisessa vaiheessa. Muiden koodin
korjaaminen auttaa hahmottamaan, mitkä ovat hyviä käytäntöjä ja mitä hyötyä on
siitä, että koodin ylläpidettävyys on hyvä.

Omalle koodille tulee helposti sokeaksi, jolloin tietyn normiston avulla
läpikäynti paljastaa ongelmakohtia. Myös aika auttaa näkemään oman koodin
uudessa valossa. Oman koodin tarkastelu pidemmän ajan päästä paljastaa itselleen
kehityksen, joka on tullut ajan saatossa. Mutta on tärkeä palata vanhaan koodiin
ja etsiä ongelmakohdat, jotta ne on helpompi tiedostaa ja korjata välittömästi
jatkossa vastaantullessa.

Omaa koodia ei tarvitse hävetä vaan siitä tulee oppia. Jokainen on joskus
kirjoittanut huonoa koodia, yleensä tiedostamatta. Siksi on tärkeä oppia
tunnistamaan huono koodi hyvästä koodista.

\subsection{Oikeanlaiset videot, kirjallinen materiaali ja tehtävät}

Opiskelijat jaksavat harvoin katsoa pitkiä videoita luennoista. Internetissä
yleistynyt formaatti on ns. screen cast. Tämän tyyppiset videot ovat enintään
kymmenen minuutin mittaisia ja sisältävät muutaman käsiteltävän asian. Nämä
asiat käsitellään sopivalla tasolla, jotta tieto on mahdollista sisäistää jo
yhden katselukerran jälkeen.

Prosessi on mahdollista tuoda mukaan videoihin. Videon alkupuolella esitellään
käsiteltävä asia pelkistetysti. Kerrotaan miten se toimii ja mihin sitä
käytetään. Tämän jälkeen liitetään asia osaksi oikeeta ohjelmaa.

Yleisen rakenteen, kuten silmukan, osalta oikea ohjelma voi esimerkiksi käydä
lävitse numerolistan alkiot ja tulostaa kunkin alkion kohdalta löytyvän luvun.
Tietorakenteen, kuten ArrayList, osalta oikea ohjelma voi esimerkiksi lisätä
päivän ruokalistaan aterioita ja tulostaa, mitä tänään on ruokavaihtoehtoina.

Oikea ohjelma voi sisältää myös aikaisemmin opitun tavan asian tekoon, jolloin
ohjelma refaktoroitaisiin käyttämään uutta tapaa. Muutenkin kaikkia mainittuja
tapoja tulee viljellä sopivasti videoihin. Liika viljely taas vie keskittymisen
käsiteltävästä asiasta. Luennolla voi taas viljellä enemmän ja tehdä ensin
virheitä, koska ne kuuluu osana ohjelmointia.

On tärkeä pitää videoiden rakenne samana, jolloin opiskelijalle on selkeä,
milloin käsitellään mitäkin asiaa.

Opiskelijat lukevat harvoin materiaalia tai siitä vain mielestään oleelliset
kohdat. Tähän ohjelmointikursseilla on ratkaisuna materiaalin ja tehtävänantojen
yhdistäminen. Tehtävät on ripoteltu materiaalin sekaan, jolloin opiskelijan on
helppo löytää tehtävää koskeva kohta materiaalista, kun se löytyy tehtävän
yläpuolelta.

Pitkä kirjallinen materiaali ei kannusta ketään lukemaan sitä, joten materiaali
on syytä pitää niin lyhyenä kuin mahdollista. Käsitelty asia tulee ilmaista
lyhyesti ja ytimekkäästi. Sen liitteenä on video, jossa näytetään miten asiaa
käytetään, ja pieni esimerkkiohjelma, josta opiskelija luultavasti kopioi
ratkaisuunsa käsitellyt asiat. Jottei teoria unohdu kokonaan, niin sille on
muutaman kappaleen verran tilaa. Tämä vastaa videolla olevaa motivaatiota.

Tehtävissä voi harrastaa toistuvuutta. Viikottain laajennetaan ja refaktoroidaan
edelliseltä viikolta tuttuja ratkaisuja. Täten ongelmaan liittyvä taustatieto
vähentyy ja itse asiaan on helppo keskittyä. Tehtävien aihealueet on syytä
miettiä tarkasti, jottei opiskelijat turhaudu ns. tylsien tehtävien parissa.

Koska tehtävien runko tulee valmiina, niin ei haittaa jos ei ehtinyt viime
viikolla tehdä kaikkea. Samalla tulee tutustuttua edellisen viikon tehtävien
mallivastauksiin ja täten oppia hyviä käytäntöjä.

Tämän tyylinen lähestymistapa näyttää opiskelijoille myös inkrementaalista
kehitystä, mikä on nykyään yleistä ohjelmistoalalla.
