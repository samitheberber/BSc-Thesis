\documentclass{tktltiki}

\usepackage{lmodern} %% Neat font



\begin{document}

\author{Sami Saada}
\title{Ohjelmointiin liittyvän ongelmaratkaisuprosessin tukeminen opetuksessa}
\level{Kandidaatintutkielma}

\maketitle

\doublespacing

\faculty{Matemaattis-luonnontieteellinen}
\department{Tietojenkäsittelytieteen laitos}
\subject{Tietojenkäsittelytiede}

\keywords{Ohjelmointiprosessi}

\begin{abstract}
Tiivistelmä
\end{abstract}

\mytableofcontents

\section{Ohjelmoinnin opetuksen historia}

Ohjelmoinnin opetus on hankalaa ja siitä ei ole konsensusta, mikä on tehokkain tapa opettaa ohjelmointia \cite{Vihavainen:2011:EAM:1953163.1953196}. Tästä asiasta on väitelty usean vuosikymmenen ajan. Useat yliopistot opettavat vielä perinteisen kaavan mukaisesti.

\subsection{Peruskysymysten asettelu}

David Griesin \cite{Gries:1974:WTI:953057.810447} näkemys ohjelmoinnin opetuksesta sisältää yhä monia oleellisia asioita, jotka ovat vieläkin relevantteja opetusta kehittäessä. Griesin mukaan ensimmäisellä ohjelmointikurssilla opiskelijan tuli saada vastaukset kysymyksiin: Miten ongelmia ratkaistaan? Miten kuvataan algoritmillinen ratkaisu ongelmaan? Miten algoritmi todetaan oikeelliseksi?

Griesin ohjelmointikurssin pääteemana on rakenteellinen ohjelmointi. Hänen mukaan kurssilla tulee opettaa yleisiä ongelmanratkaisutapoja. Opiskelijat oppivat yleensä ratkaisemaan tietyn tyyppisiä ongelmia eikä yleistä ongelmanratkaisua. Opiskelijoille on tarkoistus opettaa, miten ratkaistaan ongelma, johon on algoritmillinen ratkaisu. Kurssin tarkoitus on näyttää, miten suunnitella ja toteuttaa rakenteeltaan hyviä ohjelmia.

Perinteiseen tapaan opiskelijoille kuvataan työkalut ja annetaan muutama esimerkki. Tämän jälkeen heitä pyydetään kirjoittamaan ohjelmia. Heille ei sanota melkein mitään siitä, miten heidän tulisi aloittaa. Opiskelijoiden tulee itse löytää mahdolliset toteutustavat ja keinot jäsentää omat ajatuksensa. Myöskin mysteeriksi jää se, miten päädytään hyvin jäsennettyyn, kirjoitettuun ja luettavaan ohjelmakoodiin. Griesillä on tähän hyvä analogia:

\begin{quotation}
Kaapintekokurssilla ohjaaja esittelee sahan, laudan, vasaran ja muutaman muun työkalun parin minuutin ajan. Sitten hän näyttää kauniiksi viimeistellyn kaapin ja lopulta antaa opiskelijoille pari viikkoa aikaa toteuttaa oma kaappi.

Voisit ajatella ohjaajan olevan hullu!
\end{quotation}

Ohjelmointikursilla pitäisi pystyä opettamaan yksinkertaista ongelmanratkaisua säännöin ja lukuisin esimerkein. Esimerkit koostetaan opiskelijoiden harjoituksissa tekemistä sovelluksista. Yleistä ongelmanratkaisua on mietitty melko kovasti ja se on sovellettavissa ohjelmointiin. Descartes esittää neljä sääntöä:

\begin{enumerate}
  \item Älä koskaan hyväksy mitään totena, ellei sitä ole varmasti ja perustellusti todistettu.
  \item Jaa vaikeat asiat niin moneen tosaa kuin mahdollista.
  \item Tee ensin helpot asiat ja siirry sitten vaikeampiin.
  \item Tee luetteloista niin täydelliset ja arvosteluista niin yleisiä, ettei mitään voi jättää huomiotta.
\end{enumerate}

Hyman ja Anderson täydentävät:

\begin{enumerate}
  \item Käy läpi kaikki ongelman kohdat nopeassa tahdissa useaan kertaan, kunnes muodostuu kaava, joka sisältää kaikki kohdat.
  \item Pidätä päätöstä ja älä kiirehdi johtopäätöksiin (Descartes ensimmäinen kohta)
\end{enumerate}

Nämä säännöt ovat sovellettavissa ohjelmointiin. Jos näitä sääntöjä tottelee ohjelmoidessa, ne johtavat systemaattisuuteen ja kurinalaisuuteen, mitä suurimmalla osalla ohjelmoijista ei ole. Nämä säännöt antavat aloittelijoilla ajatuksia, jotka voivat auttaa alkuun algoritmien muodostamisessa. Ongelmaratkaisu on yksilöllistä, mutta voidaan antaa sääntöjä, jotka auttavat ymmärtämään ongelmanratkaisua.

\subsection{Yksinkertaisilla askelilla eteenpäin}

Ohjelmoijien tulee tähdätä rakenteeltaan yksinkertaisiin algoritmeihin \cite{Gries:1974:WTI:953057.810447}. Ohjelman tulee koostua pienistä osista. Jokaisen palasen on oltava riittävän pieni, jotta sen voi ymmärtää. Jos pienet palaset pystyy itsenäisesti todistamaan toimiviksi ja täten koko ohjelman voi todistaa toimivaksi. Oppiessaan lisää on mahdollista törmätä parempaan tapaan toteuttaa ohjelman palanen. Tällöin siiheb on syytä palata ja toteuttaa se uudelleen paremmalla tavalla.

Kommentointi on tärkeä osa ohjelman kehitystä. Tosin yksikäsitteiset itseään kommentoivat osat voidaan jättää kommentoimatta. Toinen tärkeä osa on kaavuot, jotka toimivat ohjelmointiprosessin apuna. Nämä kummatkin ovat osa ohjelman dokumentointia. Ohjelman dokumentoinnin tulee rohkaista systemaattiseen ja rakenteelliseen ohjelmointiin, ja ohjelmoijien tulee käyttää sitä ohjelmoidessaan.

\section{Ohjelmointiprosessi}

Ohjelmointia opettaessa keskitytään yleensä kielen rakenteisiin eikä kehitysprosessiin \cite{Caspersen:2006:NPO:1176617.1176741,Vihavainen:2011:EAM:1953163.1953196}. Caspersen ja Kölling määrittelevät, että kurssien yleinen rakenne on, että esitellään ongelma ja sitä seuraa ratkaisun tarjoava ohjelma. Ainoastaan ohjelman rakenteita käydään lävitse. Opiskelijan näkökulmasta tämä näyttää, että ohjelmat muodostetaan yhdellä askeleella. Ohjelman kehitysprosessi on usein pääasiallisesti piilotettu ja todellisuus, jossa ohjelmat alkavat epäoptimeista ja osittain toteutetusta toiminnallisuudesta, jää ammattisalaisuudeksi. Esimerkit vaativat yleensä pieniä, helposti ymmärrettäviä askelia, jotka ovat melko erilaisia kuin valmiin ohjelmiston kehittämiseen vaativista tehtävät. Ongelmana tästä on, että opiskelijat ymmärtävät rakenteet, mutteivät osaa yhdistää niitä toimivaksi kokonaisuudeksi.

Ratkaisuksi Caspersen ja Kölling ehdottavat, että opiskelijoille opetetaan sellaista ohjelmistokehitysprosessia, joka ohjaa järjestetyin askelin kohti ongelman ratkaisua. Samaa on nähtävissä Griesin esittelemässä tavassa. Ohjelmistokehitystä tulee kohdella kuin prosessia, joka koostuu osista ja pienistä askelista eikä isosta, monoliittisesta ratkaisusta. Jos ohjelmointiprosessia ei erikseen opeta, niin osa opiskelijoista ei pysty toteuttamaan haasteita. Osa taas kehittää oman prosessin, joka johtaa usein huonoihin lopputuloksiin. Sen lisäksi, että opiskelijat oppisivat ohjelmoimaan, tavoitteena on myös saavuttaa kehitystä koodin laadussa.

\subsection{Mañana periaate}

Mañana periaate on tämän pohjana ja periaate on seuraavanlainen:

\begin{quotation}
Kun metodia toteuttaessa toivot, että sinulla olisi tietynlainen apumetodi, kirjoita koodisi, kuten sinulla olisi sellainen ja toteuta apumetodi myöhemmin.
\end{quotation}

Mañanan käyttöönottaminen aloittelijoille:

\begin{description}
  \item[Erikoistapaussääntö] \hfill \\
  Jos koodissasi on jokin erikoistapaus, tee siitä oma metodi
  \item[Sisäkkäiset silmukat -sääntö] \hfill \\
  Erota sisällä oleva silmukka omaksi metodiksi
  \item[Koodin toisto -sääntö] \hfill \\
  Jos kirjoitat saman koodilohkot kahdesti, tee siitä oma metodi
  \item[Vaikean ongelman -sääntö] \hfill \\
  Jos et keksi ongelmaa heti ratkaisua, eriytä se omaksi metodiksi
  \item[Raskaan toiminnallisuuden -sääntö] \hfill \\
  Jos lauseesta tulee pitkä tai monimutkainen, eriytä se omaksi metodiksi
\end{description}

Mañana periaatteessa syntyneitä metodeja ei tarvitse toteuttaa samantien.

\bibliographystyle{tktl}
\bibliography{essee}

\lastpage

\end{document}
