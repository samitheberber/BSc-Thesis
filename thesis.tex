\documentclass{tktltiki}
\usepackage{lmodern}
\usepackage{url}

\author{Sami Saada}
\title{Ohjelmointiin liittyvän ongelmaratkaisuprosessin tukeminen opetuksessa}
\level{Kandidaatintutkielma}
\faculty{Matemaattis-luonnontieteellinen}
\department{Tietojenkäsittelytieteen laitos}
\subject{Tietojenkäsittelytiede}
\keywords{Ohjelmointiprosessi, opetus, hyvä koodi, videot, kisällioppiminen,
pajaopetus}

\numberofpagesinformation{\numberofpages\ sivua}

\begin{document}

\maketitle

\doublespacing

\begin{abstract}
Ohjelmoinnin opetuksessa on pitkään mietitty ongelmanratkaisuprosessia ja miten
sen voi tuoda mukaan osaksi ohjelmoinnin opetusta. Tämä tutkielma käy aluksi
läpi miten ohjelmointia on opetettu ajan saatossa, selvittää mitä hyviä ja
huonoja puolia eri tavoissa on ollut. Käydään lävitse yleistä
ongelmanratkaisuprosessia ja sen soveltamista ohjelmointiin. Miten
kisällioppiminen toimii mahdollistajana hyvälle ohjelmoinnin opetukselle.
Millaista ohjelmoinnin opetuksen tulee olla käytetyn materiaalin ja sisällön
osalta.
\end{abstract}

\mytableofcontents

\section{Johdanto}

Perinteisesti ohjelmointia on opetettu kirjallista materiaalia näyttämällä ja
muutamalla materiaaliin pohjautuvalla yksinkertaisella ohjelmointitehtävällä.
Tällä tavalla järjestetyllä kurssilla opiskelijat oppivat lähinnä ratkaisemaan
tietyntyyppisiä tehtäviä eikä yleistä ongelmanratkaisua
\cite{Gries:1974:WTI:953057.810447}.

Aloittelijoille suunnatut tekstit käsittelevät yleisten ohjelmointiin liittyvien
rakenteiden sijaan ohjelmointikielen rakenteisiin
\cite{Caspersen:2006:NPO:1176617.1176741, Vihavainen:2011:EAM:1953163.1953196}.
Opiskelijoille esitetään ongelma ja sen ratkaiseva valmis ohjelma. Ohjelman
rakenteet käydään lävitse materiaalissa tai luennolla. Tällöin opiskelijoille
syntyy kuva, että ohjelmat syntyvät yhdellä askeleella sopivia rakenteita
yhdistelemällä ja ohjelman todellinen kehitysprosessi ei ilmene opiskelijalle.
Seurauksena tästä on se, että opiskelijat ymmärtävät eri rakenteet, mutta eivät
osaa luoda niitä yhdistelemällä toimivaa ohjelmaa.

Esimerkkiohjelmat ovat yleensä pieniä ja koostuvat yksinkertaisista askelista,
eikä tämä vastaa todellisuutta. Lisäksi ne opettavat vain kielen rakenteita
\cite{Astrachan:1995:ACA:199691.199694}. Merkittävän kokoiset tehtävät vaativat
ulkoista tukea \cite{Kolling:2008}. Ulkoisena tukena tarkoitan kehitysympäristöä
(IDE). Tavallisesti tällainen ympäristö ei anna hyvää tukea opettamis- ja
oppimisprosessille. Javan ottaminen opetuskieleksi toi hyviä ja huonoja asioita
mukanaan. Sille on hyviä kehitysympäristöjä ja se on suhteellisen vakaa kieli.
Kuitenkin se lisäsi enemmän keskittymistä kielen rakenteisiin.

Ohjelmointiprosessin olessa salassa, opiskelijat eivät pysty ratkaisemaan
annettuja haasteita tai heille kehittyy oma prosessi
\cite{Caspersen:2006:NPO:1176617.1176741}. Yleensä itsekehittynyt prosessi
johtaa huonoihin lopputuloksiin. Ohjelmoinnin oppiminen ei saa olla ainoana
tavoitteena vaan myös koodin laadun kehitys on tärkeää.

Olemassa olevan koodin lukeminen ja muokkaaminen ovat yleisiä tehtäviä kelle
tahansa ohjelmoijalle \cite{Kolling:2008}. Oppimisen kannalta on suotavaa käydä
lävitse muiden kirjottamaa koodia varhaisessa vaiheessa. Muiden koodin
korjaaminen auttaa hahmottamaan, mitkä ovat hyviä käytäntöjä ja mitä hyötyä on
ylläpidettävästä koodista.

\section{Ongelmanratkaisuprosessi}

Tässä osiossa esittellään ensin yleinen ongelmanratkaisuprosessi, minkä jälkeen
sitä sovelletaan ohjelmointiin.

\subsection{Yleinen ongelmanratkaisuprosessi}

Yleistä ongelmanratkaisuprosessia on tutkittu pitkään ja se on sovellettavissa
monelle alalle, myös ohjelmointiin. Descartes, Hyman ja Anderson ovat
muodostaneet yleiselle ongelmanratkaisuprosessille rungon, jossa on
seuraavanlaisia piirteitä \cite{Gries:1974:WTI:953057.810447}.

Mitään asiaa ei tule koskaan hyväksyä totena ellei sille ole varmaa ja
perusteltua tietoa. Jos asian paikkaansapitävyyttä ei tiedetä, sen ei pidä
olettaa olevan totta. Tästä on apua varsinkin silloin, kun etsitään tietoa
ongelmaan ja sen ratkaisutavoista. Lähteiden taustat tulee tarkistaa huolella
ennen kuin aletaan soveltaa saatua tietoa. Korjaaminen voi olla haastavaa, jos
on sovellettu virheellistä tietoa.

Vaikeat asiat tulee jakaa niin moneen osaan kuin mahdollista. Pieniä asioita on
helpompi hallita kuin isoja. Tämän seurauksena yhden ison ongelman sijaan on
monta pientä ongelmaa, jotka voi olla hyvinkin helppoja ratkaista. Voi olla
hyvinkin mahdollista, että pienemmät ongelmat ovat entuudestaan tuttuja, joten
niihin löytyy nopeasti ratkaisu.

Toiset ongelmat ovat haastavampia kuin toiset. Kannattaa ensin ratkaista helpot
ongelmat ja sitten vaikeammat. Helpot ongelmat voivat olla nopeita ratkaista ja
niiden ratkaiseminen tuo itsevarmuutta vaikeampien ongelmien parissa.

Johtopäätöksessä tulee ilmaista kaikki ongelmaan liittyvät seikat sekä sen tulee
olla niin yleispätevä, ettei sitä voi sivuuttaa. Kaikki ongelmaan liittyvät
asiat tulee esitellä ja ne tulee ilmaista siten, ettei niitä voi tulkita väärin.

Aluksi on tärkeä käydä kaikki ongelman yksityiskohdat nopeasti moneen kertaan
lävitse, jotta kokonaiskuva ongelmasta selviää. Esimerkiksi ruuanvalmistuksessa
on syytä käydä kaikki vaiheet lävitse, jotta tiedetään, mitä aineksia ja
välineitä tarvitsee ostaa tai ottaa esille. Tämän jälkeen ongelma on
yksinkertaista jakaa moneen osaan. Ruokareseptit on jaettu pienempiin
yksinkertaisiin vaiheisiin.

Esimerkiksi tutkielman kirjoittamiseen voidaan soveltaa yleistä
ongelmanratkaisuprosessia. Ensin on saatava yleiskuva aiheesta, joka voisi olla
esimerkiksi, tietokonepelien vaikutus opiskeluun. Kun on yleiskuva, niin siitä
voi kaivaa esiin pääpiirteet, joita ovat esimerkiksi millaisia
tietokonepelityyppejä on. Pääpiirteet voidaan jakaa pienempiin alilukuihin,
kuten väkivaltaa sisältävät pelit ja opetuspelit. Lopuksi tehdään yhteenveto,
jossa kerrataan ongelmakohdat ja tutkimustulokset. Tässä on ensin jaettu isommat
kokonaisuudet pienempiin ja vielä pienempiin osiin, jotka on sitten kirjoitettu
auki.

\subsection{Ohjelmointiin soveltaminen}

Yleinen ongelmanratkaisuprosessi on sovellettavissa ohjelmointiin ja sen
seuraaminen tuo ohjelmointiin määrätietoisuutta ja selkeyttä.

\subsubsection{Griesin analogia}

Gries esittelee hyvän analogian ohjelmointiin liittyvästä
ongelmanratkaisuprosessista \cite{Gries:1974:WTI:953057.810447}:

\begin{quotation}
``Kaapinvalmistuskurssilla ohjaaja esittelee sahan, työtason, vasaran ja
muutaman muun työkalun muutamassa minuutissa. Tämän jälkeen hän näyttää
kauniin valmiin kaapin ja antaa opiskelijoille pari viikkoa aikaa tehdä oma
kaappi. Ajattelisit ohjaajan olevan hullu!''
\end{quotation}

Analogia kuvaa hyvin ohjelmoinnin opetusta. Opiskelijoille annetaan työkalut ja
näytetään pari esimerkkiohjelmaa, mutta ei kerrota, miten esimerkkiohjelmat on
toteutettu ja miksi. Tämän tyyppinen opetus rikkoo yleisen
ongelmanratkaisuprosessin pääpiirteitä.

Ongelmalähtöisessä opetuksessa opiskelijoille tulee kuvata ensin ongelma ja
sitten näyttää miten sitä lähdetään pilkkomaan pienempiin ongelmapalasiin. Tämän
jälkeen tulee kertoa, mitä ohjelmointirakenteita käytetään minkäkin osaongelman
kohdalla. Lopulta opiskelijalle on selvää, mistä ohjelma koostuu ja miksi
mikäkin rakenne on kuhunkin kohtaan valittu.

\subsubsection{Mañana-periaate}

Tämä luku käsittelee Mañana-periaatetta, joka pohjautuu lähteeseen
\cite{Caspersen:2006:NPO:1176617.1176741}.

Yksi tapa soveltaa yleistä ongelmanratkaisuprosessia on jakaa vaikeammat asiat
aliohjelmakutsujen taakse ja toteuttaa varsinaiset aliohjelmat myöhemmin. Tähän
ideaan pohjautuu Mañana-periaate. Vaikea asia voi olla sellainen, johon
ohjelmoija ei keksi heti ratkaisua.

Vaikeat asiat eivät ole ainoa tilanne, jolloin tulee käyttää Mañanaa. Jos
ohjelmassa tulee vastaan jokin erikoistapaus, sille tulee tehdä oma aliohjelma.
Tälläinen erikoistapaus on esimerkiksi janan piirtämisessä tilanne, jossa
kummatkin päätepisteet ovat samassa koordinaatissa.

Ohjelmakoodia tulee aina välillä refaktoroida. Tällöin Mañanan mukaan
monimutkaiset koodinpätkät tulee sijoittaa omiin aliohjelmiin. Pitkät lauseet,
kuten usean vaiheen sisältävät ehtolauseet, tulee refaktoroida omiin
aliohjelmiin, jolloin koodi säilyy lyhyenä ja helpommin luettavana.

Jos jokin koodipätkä toistuu useaan kertaan, se tulee eriyttää aliohjelmaan.
Tällöin, jos kyseiseen pätkään tulee muutos, riittää, että ohjelmoija tekee
tarvittavan muutoksen vain yhteen paikkaan ja koko ohjelma toimii halutulla
tavalla. Sisäkkäisten silmukoiden kohdalla sisäsilmukan koodi tulee sijoittaa
omaan aliohjelmaan.

Mañana-periaatetta noudattaessa koodi säilyy helposti luettavana ja ohjelman
oikeellisuus on helppo todentaa. Suuremmat kokonaisuudet koostuvat pienistä
osista, joita on helppo tarkastella ja niiden toimivuus todentaa. Täten
suuremmat kokonaisuudet pystytään osoittamaan toimiviksi.

\subsubsection{Järjestelmällinen ohjelman kasaaminen}

\label{järjestelmällisyyteen pohjautuva ongelmanratkaisuprosessi}

Järjestelmällisen ohjelman kasaamisen näyttäminen opiskelijoille ohjaa
systemaattiseen ajatteluun ja noudattaa yleistä ongelmanratkaisuprosessia.
Tanskassa on tutkittu tätä varsin hyvin tuloksin
\cite{Caspersen:2006:NPO:1176617.1176741}.

Opiskelijoille annetaan rajapinta, joka ratkaisun pitää toteuttaa. Rajapinnan
pohjalta he luovat luokan ja metodirungot. Metodirunkojen tulee olla
mahdollisimman yksinkertaisia eli niiden pitää sisältää tarvittavat tiedot,
jotta koodi menee kääntäjästä lävitse. Jos metodi ei palauta mitään, se on
tyhjä, ja muussa tapauksessa se palauttaa tyhjää \texttt{null} tai \texttt{0}
tietotyypistä riippuen.

Tyhjän luokkarakenteen jälkeen luodaan testit. Testit ovat joko yksikkötestejä
tai aloittelijoiden tapauksessa pääohjelmaan kirjotettuja kutsuja, jotka
määrittelevät luokan toiminnan. Testien on mentävä kääntäjästä lävitse vaikkei
niiden ajaminen tuota oikeaa tulosta.

Testien jälkeen voidaan miettiä vaihtoehtoisia esitystapoja ilmentymämuuttujien
avulla. On tärkeää painottaa, että toteutusmalleja on useita erilaisia, joten
tässä vaiheessa on löydettävä ainakin kaksi erilaista vaihtoehtoa. Esimerkiksi
päivämäärän voi tallentaa päivinä, kuukausina ja vuosina tai päivinä ajanlaskun
alusta. Kun vaihtoehdot ovat selvillä, arvioidaan metodien toteutuksien
haastavuutta eri vaihtoehdoille. Pelkkien kuluneiden päivien antaminen on
todella yksinkertaista toteuttaa, kun taas päivissä, kuukausissa ja vuosissa
ilmoitettuna tarvitaan erilaisia tarkistuksia, jotta tieto on oikein.
Päivämäärän tulostaminen on taas helppoa, kun eri osat on tallennettu erikseen.

Eri esitysvaihtoehdoista valitaan keskimäärin helpoin toteuttaa. Luodaan
ilmentymämuuttujat alustuksineen ja määritellään raja-arvot, kuten kuukausissa
mahdolliset arvot ovat yhden ja kahdentoista välillä. Tämän jälkeen toteutetaan
metodit. Metodien toteuttamiseen Caspersen on selvittänyt hyvän algoritmin:

Metodien toteuttamisella ei ole järjestystä, mutta on suositeltavaa toteuttaa
ensin helpoin ja siirtyä sitten seuraavaksi helpompaan. Metodi on valmis, kun
siinä ei ole mitään refaktoroitavaa. On tärkeää pitää vain yksi metodi
kerrallaan työn alla, sillä siten mahdolliset ongelmat rajautuvat kyseiseen
metodiin.

\subsection{Hyvä koodi}

Hyvällä koodilla (clean code) on useita määritelmiä, mutta pääpiirteet ovat
samoja \cite{Martin:2008:CCH:1388398}. Ohjelmointiin liittyvien
ongelmanratkaisuprosessien tuloksena syntyvä koodi on yleensä hyvää koodia.

Hyvän koodin kirjoittaminen on tärkeä asia ohjelmoinnissa, sillä huono koodi
hidastaa kehitystä ja mahdollistaa virheiden syntymisen. Yleinen virhe on
todeta, että korjaa huonon koodin myöhemmin, sillä yleensä myöhemmin tarkoittaa
ei koskaan. Tällöin huono koodi jää elämään.

Koodin luettavuus on tärkeä. Logiikan tulee olla suoraviivaista, jolloin virheet
eivät huku epäselvän logiikan sekaan. Hyvää koodia on mieluisa lukea ja siitä
saa hyvän kuvan ohjelman toiminnallisuudesta. Myös muidenkin kuin alkuperäisen
kehittäjän tulee pystyä ymmärtämään koodi.

Nimeämiskäytännön tulee olla selkeä ja asioilla tulee olla selkeät nimet.
Koodissa ei tule olla toistoa. Eri asioiden, kuten luokkien, metodien,
funktioiden ja samankaltasuuden, määrä tulee minimoida.

Riippuvuuksien tulee olla vähäisiä, mikä helpottaa ylläpitämistä. Riippuvuuksien
pitää olla selkeästi määriteltyjä ja niiden pitää tarjota selkeä, pieni
käyttörajapinta.

Yksikkötestit on tärkeä osa hyvää koodia, sillä ne kertovat koodin
toimivuudesta. Testattua koodia on helppo parantaa ja testit tuovat varmuutta,
että muutettu koodi tekee myös ne asiat, joita se alunperin teki.
Virheidenkäsittelyn tulee pysyä samana sovelluksen eri osissa. Testit
määrittävät koodin ulkoisen rajapinnan, joten niillä on helppo valvoa
virheidenkäsittelyä.

Testivetoinen kehitys (lyhyesti TDD) on yleinen tapa kirjoittaa koodia. TDD:ssä
ensin kirjoitetaan testi, jonka jälkeen testin toteuttava koodi. Tällä tavoin
meneteltynä koodista tulee hyvää ja testattua.

Suorituskyvyn tulee olla hyvä, mutta se ei saa tuoda mukanaan sekavuutta
koodiin. Jotkut optimoinnit voivat olla todella rumia, joten ne on syytä jättää
pois.

Hyvä koodi on yleisesti ottaen myös kaunista koodia, joka on kuin ongelmaa
kuvaava kieli. Jokainen koodin osa tekee juuri sen asian, mikä sen on
tarkoituskin tehdä eikä yhtään enempää.

\section{Kisällioppiminen}

Extreme Apprenticeship (XA) on Helsingin yliopiston tietojenkäsittelytieteen
laitoksella lanseerattu versio ohjelmoinnin kisällioppimisesta
\cite{Vihavainen:2011:EAM:1953163.1953196}. Kisällioppimisesta on useita eri
muotoja, mutta XA:n pohjimmainen tavoite on todella paljon tehtäviä ja vähän
luentoja.

Ohjelmointia voi oppia vain ohjelmoimalla, koska ohjelmointiprosessia ei voi
oppia oppikirjasta lukemalla. Tässä XA on onnistunut todella hyvin. Tehtäviä on
tosi monta ja ne täydentävät toisiaan. Toisaalta tehtävät on valmiiksi pilkottu
pieniksi ongelmiksi, joten opiskelijat eivät pääse itse pilkkomaan alkuperäisiä
ongelmia. Heidän on myöskin vaikea lähteä pilkkomaan pienempiä ongelmia vielä
pienemmiksi.

Pienien tehtävien taustalla on yksikkötestauksen helpottaminen. Tehtäviä varten
on palautusautomaatti, joka pohjautuu yksikkötesteihin. Yksikkötestit tuovat
opiskelijoille selkeän palautteen, toimiiko heidän ohjelmansa, ja
tutkimustuloksia niiden hyödyllisyydestä opetuksessa on esitetty
\cite{Bennedsen:2008}.

Luentojen suhteen XA on liian raju. Ainoa tilaisuus, missä opiskelijoilla on
mahdollisuus nähdä ja kysyä ohjelmointiprosessista, kisällioppimisessa on
luento. Jos luentoja ei ole, niin opiskelijat kehittävät oman prosessinsa, joka
aikaisemmin todettiin kohtalokkaaksi.

Pajatilaisuuksissa on tärkeä seurata opiskelijoiden yleistä työskentelyä. On
tärkeä selvittää, miten päädyttiin kyseiseen ongelmatilanteeseen, koska
ongelmatilanteissa opiskelijat kysyvät apua. Myös toimivissa tilanteissa on
tärkeä kysyä, miksi on toteutettu valitulla tavalla.

Refaktorointi on syytä ottaa esille pajatilanteessa. Tehtävien pohjautuessa
toisiinsa, on tärkeä refaktoroida välillä. Tällöin uusien ominaisuuksien tuonti
vanhojen rinnalle on puhdasta ja vaivatonta.

Kaksisuuntainen palaute on mahdollista kisälliopetuksessa. Opiskelija saa
vastauksen kysymyksiinsä ja ohjaajat saavat tietoa eri asioiden sisäistämisestä.
Tämä on erityisen tärkeä asia luentoja muodostaessa, sillä silloin voi
palautteen pohjalta muodostaa entistä antoisampia luentoja.

\section{Opetukseen sisällyttäminen}

On erityisen tärkeää tuoda ongelmanratkaisuprosessi esille opetuksessa.
Oppikirjoista ei voi opetella tätä vaan tarvitaan vaihtoehtoisia
oppimateriaaleja, joita käsitellään tässä luvussa. Koko luku perustuu lähteeseen
\cite{Bennedsen:2008}.

Liitutaulun käyttö on yleistä matematiikan opetuksessa ja sitä voidaan hyödyntää
samoin ohjelmoinnin opetuksessa. Opiskelijat ovat mukana kehityksessä, joten he
voivat kommentoida ja ohjata toteutusta. Ongelmana liitutaulun käytössä on se,
että tila on todella rajallista, joten suurempia ohjelmia sillä ei voi esittää.

Piirtoheittimen käyttö ratkaisee tilan puuteen, mutta valmiiksi tehtyihin
kalvoihin opiskelijat eivät pääse vaikuttamaan. Ratkaisun esitystahti on
haastava pitää sopivana, jotta opiskelijat ymmärtävät varmasti, mitä missäkin
tehtiin ja miksi. Videotykiltä esitetyissä esimerkeissä on täsmälleen sama
ongelma, joten uusi teknologia ei ratkaise ongelmaa tällä tavoin.

Teknologia mahdollistaa tarvittavat olosuhteet. Liitutaulun käytössä opiskelijat
otetaan mukaan ja piirtoheitinkalvoihin saa tarpeeksi isoja kokonaisuuksia.
Tietokone ja videotykki mahdollistavat kehitystyökalujen käytön luennolla,
jolloin opiskelijat näkevät oikeanlaista ohjelmointia ja voivat itse vaikuttaa
lopputulokseen. Kuitenkin luento kestää vain määrätyn ajan, joten kovin
monimutkaisia ohjelmia ei voida esittää. Lisäksi opiskelija näkee
ohjelmointiprosessin vain kerran eikä hän voi palata siihen kuten esimerkiksi
kirjan esimerkkiin, jonka voi lukea monta kertaa.

Onneksi nykyään luennot on mahdollista nauhoittaa, jolloin opiskelijat voivat
katsoa luentovideon moneen kertaan, kunnes ymmärtävät miten ohjelmointiprosessi
toimi luennolla esitetyssä ongelmassa. Tämän jälkeen heidän on helppo palata
tarkistamaan kehitysprosessin eteneminen, kun tehtäviä tehdessä tarvitsevat
apua. Eri ihmisillä kestää eri aika sisäistää luennon asia, joten
kertausmahdollisuus on eduksi.

On tärkeää, että videoissa on sisällysluettelo, jolloin opiskelijat löytävät
helposti tarvitsemansa kohdan videosta. Mitä helpommin opiskelija saa videoilta
etsimänsä tiedon, sitä vähemmän kirjallisesta materiaalista on tarve etsiä
tiettyä asiaa.

Liian käsikirjoitetut luennot ovat huonoja, sillä silloin tilanne ei vastaa
todellisuutta ja opiskelijat voivat tylsistyä. Sopiva määrä improvisointia ja
mahdolliset virheet ovat hyväksi, koska silloin nähdään, että ohjelmoidessa on
luontaista tehdä välillä virheitä. Myös kääntäjän antamia virheilmoituksia on
syytä käsitellä, jotta ymmärretään, mistä virheet johtuvat.

Luennoilla on tärkeä näyttää myös refaktorointia. Aluksi tuetettu ratkaisu tekee
asiansa, mutta se ei välttämättä ole hyvää koodia. Refaktorointi on yleinen osa
ohjelmointia, joten se tulee tuoda siten myös esille ja näyttää miten
refaktoroidaan.

Olemassa olevan koodin lukeminen ja muokkaaminen ovat tavallisimpia tehtäviä
kelle tahansa ohjelmoijalle \cite{Kolling:2008}. Oppimisen kannalta on suotavaa
käydä lävitse muiden kirjottamaa koodia varhaisessa vaiheessa. Muiden koodin
korjaaminen auttaa hahmottamaan, mitkä ovat hyviä käytäntöjä ja mitä hyötyä on
siitä, että koodin ylläpidettävyys on hyvä.

\section{Yhteenveto}

Ohjelmoinnin opetuksessa on aiemmin keskitytty kielen rakenteisiin
ohjelmarakenteiden ja ongelmanratkaisuprosessin sijaan. Kisällioppiminen tuo
tähän parannusta, mutta ei yksinään ratkaise kaikkea. Oikeanlaisilla tehtävillä
ja ohjauksella pajatilanteesta saa paljon irti.

Hyvä koodi on tärkeä osa ohjelmointia. Se sisältää piirteitä eri ohjelmointiin
liittyvistä ongelmanratkaisuprosesseista. Tämä on toisiaan ruokkiva ympyrä,
josta on syytä pitää kiinni.

Luentojen rooli on edelleen todella tärkeä. Luennoilla ei riitä, että esitellään
ohjelmointirakenteet ja valmis ratkaisu. On tärkeää tuoda ilmi koko prosessi
ongelman analysoinnista oikeiden rakenteiden valitsemiseen ja valmiin ohjelman
saamiseen usean refaktorointikierroksen jälkeen valmiiksi.

Videot ovat tärkeä lisä materiaaliin. Niihin voi palata tehtäviä tehdessä ja
niissä saa korostettua ja näytettyä asioita, joihin kirjallinen materiaali ei
kykene.

Kaksisuuntainen palaute on tärkeää opintojen suunnittelussa. Opiskelija saa
palautetta edistymisestään ja ohjaaja saa tiedon asioiden sisäistämisestä.
Luentoja suunnitellessa tämä tieto on tärkeä, jotta voidaan keskittyä epäselviin
asioihin.


\bibliographystyle{tktl}
\bibliography{thesis}

\lastpage

\end{document}
