\documentclass[finnish]{beamer}
\usetheme[ml]{HY}
\usepackage{lmodern}
\usepackage[utf8]{inputenc}
\usepackage{url}
\usepackage{babel}

\author{Sami Saada}
\title{Ohjelmointiin liittyvän ongelmaratkaisuprosessin tukeminen opetuksessa}
\institute{Tietojenkäsittelytieteen laitos}

\begin{document}

\HyTitle

\frame
{
  \frametitle{Sisältö}
  \tableofcontents
}

\section{Mitä olen saanut selville?}

\frame
{
  \frametitle{Yleinen ongelmanratkaisuprosessi}

  Descartes, Hyman ja Anderson
}

\frame
{
  \frametitle{Ongelmanratkaisuprosessin soveltaminen ohjelmointiin}

  Mañana-periaate
}

\frame
{
  \frametitle{Ongelmanratkaisuprosessin soveltaminen ohjelmointiin}

  Järjestelmällinen ohjelman kasaaminen
}

\frame
{
  \frametitle{Ongelmanratkaisuprosessin sisällyttäminen opetukseen}

  Liitutaulun käyttö
}

\frame
{
  \frametitle{Ongelmanratkaisuprosessin sisällyttäminen opetukseen}

  Piirtoheittimen käyttö
}

\frame
{
  \frametitle{Ongelmanratkaisuprosessin sisällyttäminen opetukseen}

  Videotykki ja IDE
}

\frame
{
  \frametitle{Ongelmanratkaisuprosessin sisällyttäminen opetukseen}

  Luentojen nauhoittaminen
}

\frame
{
  \frametitle{Kisällioppimisen toimii mahdollistajana}

  Tekemisen meininki
}

\frame
{
  \frametitle{Kisällioppimisen toimii mahdollistajana}

  Ohjaajalta apua
}

\frame
{
  \frametitle{Kisällioppimisen toimii mahdollistajana}

  Kaksisuuntainen palaute
}

\section{Mihin ajattelin syventyä?}

\frame
{
  \frametitle{Parannuksia nykyiseen kisällioppimiseen}

  Miten videoita voi parantaa?
}

\frame
{
  \frametitle{Parannuksia nykyiseen kisällioppimiseen}

  Miten pajaohjausta voi parantaa
}

\frame
{
  \frametitle{Mitä muuta?}

  Kiitos!
}

\end{document}
