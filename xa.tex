\section{Kisällioppiminen}

Extreme Apprenticeship (XA) on Helsingin yliopiston tietojenkäsittelytieteen
laitoksella lanseerattu versio ohjelmoinnin kisällioppimisesta
\cite{Vihavainen:2011:EAM:1953163.1953196}. Kisällioppimisesta on useita eri
muotoja, mutta XA:n pohjimmainen tavoite on todella paljon tehtäviä ja vähän
luentoja.

Ohjelmointia voi oppia vain ohjelmoimalla, koska ohjelmointiprosessia ei voi
oppia oppikirjasta lukemalla. Tässä XA on onnistunut todella hyvin. Tehtäviä on
tosi monta ja ne täydentävät toisiaan. Toisaalta tehtävät on valmiiksi pilkottu
pieniksi ongelmiksi, joten opiskelijat eivät pääse itse pilkkomaan alkuperäisiä
ongelmia. Heidän on myöskin vaikea lähteä pilkkomaan pienempiä ongelmia vielä
pienemmiksi.

Pienien tehtävien taustalla on yksikkötestauksen helpottaminen. Tehtäviä varten
on palautusautomaatti, joka pohjautuu yksikkötesteihin. Yksikkötestit tuovat
opiskelijoille selkeän palautteen, toimiiko heidän ohjelmansa, ja
tutkimustuloksia niiden hyödyllisyydestä opetuksessa on esitetty
\cite{Bennedsen:2008}.

Luentojen suhteen XA on liian raju. Ainoa tilaisuus, missä opiskelijoilla on
mahdollisuus nähdä ja kysyä ohjelmointiprosessista, kisällioppimisessa on
luento. Jos luentoja ei ole, niin opiskelijat kehittävät oman prosessinsa, joka
aikaisemmin todettiin kohtalokkaaksi.

Pajatilaisuuksissa on tärkeä seurata opiskelijoiden yleistä työskentelyä. On
tärkeä selvittää, miten päädyttiin kyseiseen ongelmatilanteeseen, koska
ongelmatilanteissa opiskelijat kysyvät apua. Myös toimivissa tilanteissa on
tärkeä kysyä, miksi on toteutettu valitulla tavalla.

Refaktorointi on syytä ottaa esille pajatilanteessa. Tehtävien pohjautuessa
toisiinsa, on tärkeä refaktoroida välillä. Tällöin uusien ominaisuuksien tuonti
vanhojen rinnalle on puhdasta ja vaivatonta.
