\section{Kisällioppiminen}

Extreme Apprenticeship (XA) on Helsingin yliopiston tietojenkäsittelytieteen
laitoksella kehitelty versio ohjelmoinnin kisällioppimisesta
\cite{Vihavainen:2011:EAM:1953163.1953196}. Kisällioppimisesta on useita eri
muotoja, mutta XA:n pohjimmainen tavoite on todella paljon tehtäviä ja vähän
luentoja. % TODO: selkeytä tutkielmassa

Ohjelmointia voi oppia vain ohjelmoimalla, eikä ohjelmointiprosessia voi oppia
oppikirjasta lukemalla. Tässä XA on onnistunut todella hyvin. Tehtäviä on tosi
monta ja ne täydentävät toisiaan. Toisaalta tehtävät on valmiiksi pilkottu
pieniksi ongelmiksi, joten opiskelijat eivät pääse itse pilkkomaan alkuperäisiä
ongelmia. Heidän on myöskin vaikea lähteä pilkkomaan pienempiä ongelmia vielä
pienemmiksi. % TODO: selkeytä tutkielmassa: miksi?

Pienien tehtävien taustalla on yksikkötestauksen helpottaminen. Tehtäviä varten
on palautusautomaatti, joka pohjautuu yksikkötesteihin. Yksikkötestit tuovat
opiskelijoille palautteen, toimiiko heidän ohjelmansa. Tutkimustuloksia niiden
hyödyllisyydestä opetuksessa on esitetty \cite{Bennedsen:2008}.
% TODO: millaisia tuloksia?

Luentojen suhteen XA on liian jyrkkä. Kisällioppimisessa ainoa tilaisuus, jossa
opiskelijoilla on mahdollisuus nähdä ja kysyä varsinaisesta
ohjelmointiprosessista, on luento. Jos luentoja ei ole, opiskelijat kehittävät
oman prosessinsa, mikä aikaisemmin todettiin kohtalokkaaksi.

% TODO: vaihtoehtoiset opetustavat

Kisällioppimisessa järjestetään pajatilaisuuksia, joissa opiskelijat tekevät
tehtäviä ja voivat kysellä ohjaajilta apua. Pajatilaisuuksissa on tärkeä seurata
opiskelijoiden työskentelyä. Kun ongelmatilanteissa opiskelijat kysyvät apua, on
tärkeä selvittää, miten opiskelija päätyi kyseiseen ongelmatilanteeseen. Täten
ohjaaja pystyy antamaan opiskelijalle suoraa palautetta, jonka pohjalta
opiskelija voi jatkossa välttää samankaltaisen ongelmatilateen. Myös toimivissa
tilanteissa on tärkeä kysyä, miksi ratkaisu on toteutettu valitulla tavalla.
Tällä kysymysellä on tarkoitus herättää ajatus ratkaisun selkeydestä ja
refaktoroinnista.

Refaktorointi on syytä ottaa esille pajatilanteessa. Tehtävien pohjautuessa
toisiinsa, on tärkeä refaktoroida välillä. Tällöin uusien ominaisuuksien tuonti
vanhojen rinnalle on helpompaa.

Kaksisuuntainen palaute on mahdollista kisälliopetuksessa, sillä luennoija on
yksi pajaohjaajista. Opiskelija saa vastauksen kysymyksiinsä ja ohjaajat saavat
tietoa eri asioiden sisäistämisestä. Tämä on erityisen tärkeä asia luentoja
muodostaessa, sillä silloin voi palautteen pohjalta muodostaa entistä
antoisampia luentoja.
