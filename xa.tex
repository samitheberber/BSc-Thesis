\section{Kisällioppiminen}

Extreme Apprenticeship (XA) on Helsingin yliopiston tietojenkäsittelytieteen
laitoksella kehitelty versio ohjelmoinnin kisällioppimisesta
\cite{Vihavainen:2011:EAM:1953163.1953196}. Ohjelmointi on suuressa osassa
aktiiviseen oppimiseen pohjautuvissa metodeissa. Kisällioppimisessa pääpaino on
prosessissa eikä lopullisessa tuotteessa.

Kisällioppimisessa on kolme vaihetta: mallintaminen, oppimisen oikeaoppinen
tukeminen ja häivytys. Mallintamisenvaiheessa opettaja antaa opiskelijalle
konseptuaalisen mallin prosessista, eli esimerkein näyttää ohjelman tuottamisen
alusta loppuun. Opettaja pohtii tehtävän kaikki vaiheet ääneen, joten
opiskelijalle syntyy kuva, miten opettaja ratkaisi tehtävän askel askeleelta.

Oppimisen oikeaoppinen tukeminen on pajassa tehtävien tekemistä kokeneempien
opastuksessa. Opastajat auttavat opiskelijoita eteenpäin antamalla tarpeeksi
vihjeitä, jotta opiskelija oivaltaa ratkaisun itse.

Häivytysvaiheessa opiskelija alkaa itse hallitsemaan tehtäviä, joten tukemista
voidaan vähentää. Opiskelija on sisäistänyt hyviä malleja, joita hyväksikäyttäen
hän kehittää hyviä, toimivia ratkaisuja tehtäviin.

Ohjelmointia voi oppia vain ohjelmoimalla, eikä ohjelmointiprosessia voi oppia
oppikirjasta lukemalla. Tässä XA on onnistunut todella hyvin. Tehtäviä on tosi
monta ja ne täydentävät toisiaan. Toisaalta tehtävät on valmiiksi pilkottu
pieniksi ongelmiksi, joten opiskelijat eivät pääse itse pilkkomaan alkuperäisiä
ongelmia. Heidän on myöskin vaikea lähteä pilkkomaan pienempiä ongelmia vielä
pienemmiksi. Yksittäisen metodin toteuttaminen kelpaa yleensä ratkaisuksi, joten
tehtävän suorittaminen ei vaadi pilkkomista, täten se helposti sivuutetaan.

Pienien tehtävien taustalla on yksikkötestauksen helpottaminen. Tehtäviä varten
on palautusautomaatti, joka pohjautuu yksikkötesteihin. Yksikkötestit tuovat
opiskelijoille palautteen, toimiiko heidän ohjelmansa. Tutkimustuloksia niiden
hyödyllisyydestä opetuksessa on esitetty \cite{Bennedsen:2008}. Testien avulla
opiskelija tietää, mitä ohjelmalta todellisuudessa halutaan.

Puhtaassa XA:ssa halutaan minimoida luentojen määrä. Kisällioppimisessa ainoa
tilaisuus, jossa opiskelijoilla on mahdollisuus nähdä ja kysyä varsinaisesta
ohjelmointiprosessista, on luento. Jos luentoja ei ole, opiskelijat kehittävät
oman prosessinsa, mikä aikaisemmin todettiin kohtalokkaaksi.

Verkkokursseissa luennot on korvattu lyhyillä opetusvideoilla. Niiden tarkoitus
on tuoda asia esille luentoa mukailevalla tavalla. Opetusvideot sopivat myös
kurssille, jossa järjestetään luentoja. Videot ovat yksi keino lisää palata
käsiteltyyn asiaan ja siinä näkyy prosessi toisin kuin kirjoitetussa
materiaalissa.

Kisällioppimisessa järjestetään pajatilaisuuksia, joissa opiskelijat tekevät
tehtäviä ja voivat kysellä ohjaajilta apua. Pajatilaisuuksissa on tärkeä seurata
opiskelijoiden työskentelyä. Kun ongelmatilanteissa opiskelijat kysyvät apua, on
tärkeä selvittää, miten opiskelija päätyi kyseiseen ongelmatilanteeseen. Täten
ohjaaja pystyy antamaan opiskelijalle suoraa palautetta, jonka pohjalta
opiskelija voi jatkossa välttää samankaltaisen ongelmatilateen. Myös toimivissa
tilanteissa on tärkeä kysyä, miksi ratkaisu on toteutettu valitulla tavalla.
Tällä kysymysellä on tarkoitus herättää ajatus ratkaisun selkeydestä ja
refaktoroinnista.

Refaktorointi on syytä ottaa esille pajatilanteessa. Tehtävien pohjautuessa
toisiinsa, on tärkeä refaktoroida välillä. Tällöin uusien ominaisuuksien tuonti
vanhojen rinnalle on helpompaa.

Kaksisuuntainen palaute on mahdollista kisälliopetuksessa, sillä luennoija on
yksi pajaohjaajista. Opiskelija saa vastauksen kysymyksiinsä ja ohjaajat saavat
tietoa eri asioiden sisäistämisestä. Tämä on erityisen tärkeä asia luentoja
muodostaessa, sillä silloin voi palautteen pohjalta muodostaa entistä
antoisampia luentoja.
