\section{Ongelmaratkaisuprosessi}

Tässä osiossa esittelen ensin yleisen ongelmaratkaisuprosessin, minkä jälkeen
esittelen sen soveltamisen ohjelmointiin.

\subsection{Yleinen ongelmaratkaisuprosessi}

Yleistä ongelmaratkaisuprosessia on mietitty pitkään ja se on sovellettavissa
monelle alalle, myös ohjelmointiin. Descartes, Hyman ja Anderson muodostavat
hyvän rungon yleiselle ongelmaratkaisuprosessille
\cite{Gries:1974:WTI:953057.810447}.

Mitään asiaa ei tule koskaan hyväksyä totena ellei sille ole varmaa ja
perusteltua tietoa. Jos et tiedä asian paikkaansapitävyyttä, niin älä oleta sen
olevan totta. Tästä on apua varsinkin silloin, kun etsit tietoa ongelmaan ja sen
ratkaisutavoista. Tarkista huolella lähteiden taustat ennenkuin alat soveltamaan
saamaasi tietoa. Korjaaminen voi olla haastavaa virheellisen tiedon soveltamisen
jälkeen.

Vaikeat asiat tulee jakaa niin moneen osaan kuin mahdollista. Pieniä asioita on
helpompi hallita kuin isoja. Tämän seurauksena yhden ison ongelman sijaan on
monta pientä ongelmaa, jotka voi olla hyvinkin helppoja ratkaista. Voi olla
hyvinkin mahdollista, että pienemmät ongelmat ovat entuudestaan tuttuja, joten
niihin löytyy nopeasti ratkaisu.

Toiset ongelmat ovat haastavampia kuin toiset. Kannattaa ensin ratkaista helpot
ongelmat ja sitten vaikeammat. Helpot ongelmat voi olla nopeita ratkaista ja
niiden ratkaiseminen tuo itsevarmuutta vaikeampia ratkastessa.

Johtopäätökseen tulee olla kattavasti listattu ongelmaan liittyvät asiat ja
selostuksen tulee olla niin yleinen ettei sitä voi jättää huomiotta.

Aluksi on tärkeä käydä kaikki ongelman yksityiskohdat nopeasti moneen kertaan
lävitse, jotta kokonaiskuva ongelmasta selviää. Tämän jälkeen ongelma on
yksinkertainen jakaa moneen osaan.

\subsection{Ohjelmointiin soveltaminen}

\subsubsection{Mañana-periaate}

% Top-down, REM, stubit, ...

\subsection{Opetukseen sisällyttäminen}

% Videot yms. Caspersen kertonut tästä
