\section{Ongelmaratkaisuprosessi}

Tässä osiossa esittellään ensin yleinen ongelmaratkaisuprosessi, minkä jälkeen
sitä sovelletaan ohjelmointiin.

\subsection{Yleinen ongelmaratkaisuprosessi}

Yleistä ongelmaratkaisuprosessia on tutkittu pitkään ja se on sovellettavissa
monelle alalle, myös ohjelmointiin. Descartes, Hyman ja Anderson ovat
muodostaneet yleiselle ongelmaratkaisuprosessille rungon, jossa on
seuraavanlaisia piirteitä \cite{Gries:1974:WTI:953057.810447}.

Mitään asiaa ei tule koskaan hyväksyä totena ellei sille ole varmaa ja
perusteltua tietoa. Jos et tiedä asian paikkaansapitävyyttä, niin älä oleta sen
olevan totta. Tästä on apua varsinkin silloin, kun etsit tietoa ongelmaan ja sen
ratkaisutavoista. Tarkista huolella lähteiden taustat ennenkuin alat soveltamaan
saamaasi tietoa. Korjaaminen voi olla haastavaa virheellisen tiedon soveltamisen
jälkeen.

Vaikeat asiat tulee jakaa niin moneen osaan kuin mahdollista. Pieniä asioita on
helpompi hallita kuin isoja. Tämän seurauksena yhden ison ongelman sijaan on
monta pientä ongelmaa, jotka voi olla hyvinkin helppoja ratkaista. Voi olla
hyvinkin mahdollista, että pienemmät ongelmat ovat entuudestaan tuttuja, joten
niihin löytyy nopeasti ratkaisu.

Toiset ongelmat ovat haastavampia kuin toiset. Kannattaa ensin ratkaista helpot
ongelmat ja sitten vaikeammat. Helpot ongelmat voivat olla nopeita ratkaista ja
niiden ratkaiseminen tuo itsevarmuutta vaikeampien ongelmien parissa.

Johtopäätöksestä tulee selvitä kattavasti ongelmaan liittyvät erinäiset seikat
sekä sen tulee olla niin yleinen, ettei sitä tule jäättää huomiotta. Kaikki
ongelmaan liittyvät asiat tulee esitellä ja ne tulee ilmaista siten, ettei niitä
voi tulkita väärin. Tällöin johtopäätös on selkeästi esitetty ja yleisesti
ymmärrettävissä.

Aluksi on tärkeä käydä kaikki ongelman yksityiskohdat nopeasti moneen kertaan
lävitse, jotta kokonaiskuva ongelmasta selviää. Tämän jälkeen ongelma on
yksinkertaista jakaa moneen osaan.

Esimerkiksi tutkielman kirjoittamiseen voi soveltaa yleistä
ongelmaratkaisuprosessia. Ensin on saatava yleiskuva aiheesta. Kun on yleiskuva,
niin siitä voi kaivaa esiin pääpiirteet. Pääpiirteet voidaan jakaa pienempiin
alilukuihin. Lopuksi tehdään yhteenveto, jossa kerrataan ongelmakohdat ja
kerrotaan niistä yleisellä tasolla, mutta siten, että se on tulkittavissa
oikein. Tässä on ensin jaettu isommat kokonaisuudet pienempiin ja vielä
pienempiin osiin, jotka on sitten kirjoitettu auki. Lopuksi yhteenvedossa tehty
päätelmä.

Yleinen ongelmaratkaisuprosessi on sovellettavissa kaikkiin ongelmiin.

\subsection{Ohjelmointiin soveltaminen}

Yleinen ongelmaratkaisuprosessi on sovellettavissa ohjelmointiin ja sen
seuraaminen tuo ohjelmointiin määrätietoisuutta ja puhtautta.

\subsubsection{Griesin analogia}

Gries esittelee hyvän analogian ohjelmointiin liittyvästä
ongelmaratkaisuprosessista \cite{Gries:1974:WTI:953057.810447}:

\begin{quotation}
``Kaapinvalmistuskurssilla ohjaaja esittelee sahan, työtason, vasaran ja
muutaman muun työkalun muutamassa minuutissa. Tämän jälkeen hän näyttää
kauniin valmiin kaapin ja antaa opiskelijoille pari viikkoa aikaa tehdä oma
kaappi. Ajattelisit ohjaajan olevan hullu!''
\end{quotation}

Analogia kuvaa hyvin ohjelmoinnin opetusta. Opiskelijoille annetaan työkalut ja
näytetään pari esimerkkiohjelmaa, mutta ei kerrota, miten esimerkkiohjelmat on
toteutettu ja miksi. Tämän tyyppinen opetus rikkoo yleisen
ongelmaratkaisuprosessin pääpiirteitä. Opiskelijoille tulee kuvata ensin
ongelma, sitten näyttää miten sitä lähdetään pilkkomaan pienempiin
ongelmapalasiin. Tämän jälkeen tulee kertoa, mitä ohjelmointirakenteita
käytetään minkäkin osaongelman kohdalla. Lopulta opiskelijalle on selvää, mistä
ohjelma koostuu ja miksi mikäkin rakenne on kuhunkin kohtaan valittu.

\subsubsection{Mañana-periaate}

Yksi tapa soveltaa yleistä ongelmaratkaisuprosessia on jakaa vaikeammat asiat
aliohjelmakutsujen taakse ja toteuttaa varsinaiset aliohjelmat myöhemmin. Tähän
ideaan pohjautuu Mañana-periaate \cite{Caspersen:2006:NPO:1176617.1176741}.
Vaikea asia voi olla sellainen, johon ei keksi heti ratkaisua. Vaikeat asiat
eivät ole ainoa tilanne, jolloin tulee käyttää Mañanaa. Jos ohjelmassa tulee
vastaan jokin erikoistapaus, niin silloin sille tulee tehdä oma aliohjelma.

Ohjelmakoodia tulee aina välillä refaktoroida. Tällöin Mañanan mukaan
monimutkaiset koodipätkät tulee olla omissa aliohjelmissa. Pitkät lauseet tulee
refaktoroida omiin aliohjelmiin, jolloin koodi säilyy lyhyenä ja helpommin
luettavana. Jos jokin koodipätkä toistuu useaan kertaan, se tulee eriyttää
aliohjelmaan. Tällöin, jos kyseiseen pätkään tulee muutos, niin riittää, että
yhteen paikkaan tekee tarvittavan muutoksen ja koko ohjelma toimii halutulla
tavalla. Sisäkkäisten silmukoiden kohdalla sisäsilmukan koodi tulee laittaa
omaan aliohjelmaan.

Mañana-periaatetta noudattaessa koodi säilyy helposti luettavana ja ohjelman
oikeellisuus on helposti todistettavissa.

% Top-down, REM, stubit, ...

\subsection{Opetukseen sisällyttäminen}

% Videot yms. Caspersen kertonut tästä
