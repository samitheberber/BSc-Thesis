\section{Ongelmanratkaisuprosessi}

Tässä osiossa esittellään ensin yleinen ongelmanratkaisuprosessi, minkä jälkeen
sitä sovelletaan ohjelmointiin.

\subsection{Yleinen ongelmanratkaisuprosessi}

Yleistä ongelmanratkaisuprosessia on tutkittu pitkään ja se on sovellettavissa
monelle alalle, myös ohjelmointiin. Descartes, Hyman ja Anderson ovat
muodostaneet yleiselle ongelmanratkaisuprosessille rungon, jossa on
seuraavanlaisia piirteitä \cite{Gries:1974:WTI:953057.810447}.

Mitään asiaa ei tule koskaan hyväksyä totena ellei sille ole varmaa ja
perusteltua tietoa. Jos et tiedä asian paikkaansapitävyyttä, niin älä oleta sen
olevan totta. Tästä on apua varsinkin silloin, kun etsit tietoa ongelmaan ja sen
ratkaisutavoista. Tarkista huolella lähteiden taustat ennenkuin alat soveltamaan
saamaasi tietoa. Korjaaminen voi olla haastavaa virheellisen tiedon soveltamisen
jälkeen.

Vaikeat asiat tulee jakaa niin moneen osaan kuin mahdollista. Pieniä asioita on
helpompi hallita kuin isoja. Tämän seurauksena yhden ison ongelman sijaan on
monta pientä ongelmaa, jotka voi olla hyvinkin helppoja ratkaista. Voi olla
hyvinkin mahdollista, että pienemmät ongelmat ovat entuudestaan tuttuja, joten
niihin löytyy nopeasti ratkaisu.

Toiset ongelmat ovat haastavampia kuin toiset. Kannattaa ensin ratkaista helpot
ongelmat ja sitten vaikeammat. Helpot ongelmat voivat olla nopeita ratkaista ja
niiden ratkaiseminen tuo itsevarmuutta vaikeampien ongelmien parissa.

Johtopäätöksestä tulee selvitä kattavasti ongelmaan liittyvät erinäiset seikat
sekä sen tulee olla niin yleinen, ettei sitä tule jäättää huomiotta. Kaikki
ongelmaan liittyvät asiat tulee esitellä ja ne tulee ilmaista siten, ettei niitä
voi tulkita väärin. Tällöin johtopäätös on selkeästi esitetty ja yleisesti
ymmärrettävissä.

Aluksi on tärkeä käydä kaikki ongelman yksityiskohdat nopeasti moneen kertaan
lävitse, jotta kokonaiskuva ongelmasta selviää. Tämän jälkeen ongelma on
yksinkertaista jakaa moneen osaan.

Esimerkiksi tutkielman kirjoittamiseen voi soveltaa yleistä
ongelmanratkaisuprosessia. Ensin on saatava yleiskuva aiheesta. Kun on yleiskuva,
niin siitä voi kaivaa esiin pääpiirteet. Pääpiirteet voidaan jakaa pienempiin
alilukuihin. Lopuksi tehdään yhteenveto, jossa kerrataan ongelmakohdat ja
kerrotaan niistä yleisellä tasolla, mutta siten, että se on tulkittavissa
oikein. Tässä on ensin jaettu isommat kokonaisuudet pienempiin ja vielä
pienempiin osiin, jotka on sitten kirjoitettu auki. Lopuksi yhteenvedossa tehty
päätelmä.

Yleinen ongelmanratkaisuprosessi on sovellettavissa kaikkiin ongelmiin.

\subsection{Ohjelmointiin soveltaminen}

Yleinen ongelmanratkaisuprosessi on sovellettavissa ohjelmointiin ja sen
seuraaminen tuo ohjelmointiin määrätietoisuutta ja puhtautta.

\subsubsection{Griesin analogia}

Gries esittelee hyvän analogian ohjelmointiin liittyvästä
ongelmanratkaisuprosessista \cite{Gries:1974:WTI:953057.810447}:

\begin{quotation}
``Kaapinvalmistuskurssilla ohjaaja esittelee sahan, työtason, vasaran ja
muutaman muun työkalun muutamassa minuutissa. Tämän jälkeen hän näyttää
kauniin valmiin kaapin ja antaa opiskelijoille pari viikkoa aikaa tehdä oma
kaappi. Ajattelisit ohjaajan olevan hullu!''
\end{quotation}

Analogia kuvaa hyvin ohjelmoinnin opetusta. Opiskelijoille annetaan työkalut ja
näytetään pari esimerkkiohjelmaa, mutta ei kerrota, miten esimerkkiohjelmat on
toteutettu ja miksi. Tämän tyyppinen opetus rikkoo yleisen
ongelmanratkaisuprosessin pääpiirteitä. Opiskelijoille tulee kuvata ensin
ongelma, sitten näyttää miten sitä lähdetään pilkkomaan pienempiin
ongelmapalasiin. Tämän jälkeen tulee kertoa, mitä ohjelmointirakenteita
käytetään minkäkin osaongelman kohdalla. Lopulta opiskelijalle on selvää, mistä
ohjelma koostuu ja miksi mikäkin rakenne on kuhunkin kohtaan valittu.

\subsubsection{Mañana-periaate}

Yksi tapa soveltaa yleistä ongelmanratkaisuprosessia on jakaa vaikeammat asiat
aliohjelmakutsujen taakse ja toteuttaa varsinaiset aliohjelmat myöhemmin. Tähän
ideaan pohjautuu Mañana-periaate \cite{Caspersen:2006:NPO:1176617.1176741}.
Vaikea asia voi olla sellainen, johon ei keksi heti ratkaisua. Vaikeat asiat
eivät ole ainoa tilanne, jolloin tulee käyttää Mañanaa. Jos ohjelmassa tulee
vastaan jokin erikoistapaus, niin silloin sille tulee tehdä oma aliohjelma.

Ohjelmakoodia tulee aina välillä refaktoroida. Tällöin Mañanan mukaan
monimutkaiset koodipätkät tulee olla omissa aliohjelmissa. Pitkät lauseet tulee
refaktoroida omiin aliohjelmiin, jolloin koodi säilyy lyhyenä ja helpommin
luettavana. Jos jokin koodipätkä toistuu useaan kertaan, se tulee eriyttää
aliohjelmaan. Tällöin, jos kyseiseen pätkään tulee muutos, niin riittää, että
yhteen paikkaan tekee tarvittavan muutoksen ja koko ohjelma toimii halutulla
tavalla. Sisäkkäisten silmukoiden kohdalla sisäsilmukan koodi tulee laittaa
omaan aliohjelmaan.

Mañana-periaatetta noudattaessa koodi säilyy helposti luettavana ja ohjelman
oikeellisuus on helposti todistettavissa.

\subsubsection{Järjestelmällinen ohjelman kasaaminen}

Järjestelmällisen ohjelman kasaamisen näyttäminen opiskelijoille ohjaa
systemaattiseen ajatteluun ja noudattaa yleistä ongelmanratkaisuprosessia.
Tanskassa on tutkittu tätä varsin hyvin tuloksin
\cite{Caspersen:2006:NPO:1176617.1176741}.

Opiskelijoille annetaan rajapinta, jonka ratkaisun pitää toteuttaa. Rajapinnan
pohjalta he luovat luokan ja metodirungot. Metodirunkojen tulee olla
mahdollisimman yksinkertainen eli sen pitää sisältää tarvittavat tiedot, jotta
koodi menee kääntäjästä lävitse. Jos metodi ei palauta mitään, niin silloin se
on tyhjä, muutta tapauksessa se palauttaa tyhjää tietoa vastaavan asia, kuten
$null$ tai $0$ tietotyypistä riippuen.

Tyhjän luokkarakenteen jälkeen luodaan testit. Testit ovat joko yksikkötestejä
tai aloittelijoiden tapauksessa pääohjelmaan kirjotettuja kutsuja, jotka
määrittelevät luokan toiminnan. Testien on mentävä kääntäjästä lävitse vaikkei
niiden ajaminen tuota oikeaa tulosta.

Testien jälkeen voidaan miettiä esitysvaihtoehtoja ilmentymämuuttujien avulla.
On tärkeää painottaa, että toteutusmalleja on useita erilaisia, joten tässä
vaiheessa on löydettävä ainakin kaksi erilaista vaihtoehtoa. Esimerkiksi
päivämäärän voi tallentaa päivinä, kuukausina ja vuosina tai päivinä ajanlaskun
alusta. Kun eri vaihtoehdot on selvillä, mietitään metodien toteutuksien
vaikeuksia eri esitysvaihtoehdoille. Pelkkien kuluneiden päivien syöttäminen on
todella yksinkertaista toteuttaa, kun taas päivissä, kuukausissa ja vuosissa
ilmoitettuna tarvitaan erilaisia tarkistuksia, jotta tieto on oikein.
Päivämäärän tulostaminen on taas helppoa, kun eri osat on tallennettu erikseen.

Eri esitysvaihtoehdoista valitaan keskimäärin helpoin toteuttaa. Luodaan
ilmentymämuuttujat alustuksineen ja määritellään raja-arvot. Tämän jälkeen
toteutetaan metodit. Metodien toteuttamiseen Caspersen on selvittänyt hyvän
algoritmin:

Niin kauan kuin on toteuttamaton metodi, niin valitse toteuttamaton metodi ja
toteuta se. Paranna ja testaa metodia niin kauan kunnes se on valmis.

Metodien toteuttamisella ei ole järjestystä, mutta on suositeltavaa toteuttaa
ensin helpoin ja siirtyä sitten seuraavaksi helpompaan.

\subsection{Opetukseen sisällyttäminen}

On erityisen tärkeää tuoda ongelmanratkaisuprosessi esille opetuksessa.
Oppikirjoista ei voi opetella tätä vaan tarvitaan vaihtoehtoisia
oppimateriaaleja \cite{Bennedsen:2008}.

Liitutaulun käyttö on yleistä matematiikan opetuksessa ja sitä voi hyödyntää
samoin ohjelmoinnin opetuksessa. Opiskelijat ovat mukana kehityksessä, joten he
voivat kommentoida ja ohjata toteutusta. Ongelmana liitutaulun käytössä on se,
että tila on todella rajallista, joten suurempia ohjelmia sillä ei voi esittää.

Piirtoheittimen käyttö ratkaisee tilan puuteen, mutta valmiiksi tehtyihin
kalvoihin opiskelijat eivät pääse vaikuttamaan. Ratkaisun esitystahti on
haastava pitää sopivana, jotta opiskelijat ymmärtävät varmasti, mitä missäkin
tehtiin ja miksi. Videotykiltä esitetyissä esimerkeissä on täsmälleen sama
ongelma, joten uusi teknologia ei ratkaise ongelmaa tällä tavoin.

Teknologia mahdollistaa tarvittavat olosuhteet. Liitutaulun käytössä opiskelijat
otetaan mukaan ja piirtoheitinkalvoihin saa tarpeeksi isoja kokonaisuuksia.
Tietokone ja videotykki mahdollistavat kehitystyökalujen käytön luennolla,
jolloin opiskelijat näkevät oikeanlaista ohjelmointia ja voivat itse vaikuttaa
lopputulokseen. Kuitenkin luento kestää vain määrätyn ajan, joten kauhean
monimutkaisia ohjelmia ei voi esittää. Myöskin ohjelmointiprosessin näkee vain
kerran eikä siihen voi palata kuten esimerkiksi kirjan esimerkkiin, jonka voi
lukea monta kertaa.

Onneksi nykyään luennot on mahdollista nauhottaa, jolloin opiskelijat voivat
katsoa luentovideon moneen kertaan, kunnes ymmärtävät miten ohjelmointiprosessi
toimi luennolla esitetyssä ongelmassa. Tämän jälkeen heidän on helppo palata
tarkistamaan kehitysprosessin eteneminen, kun tehtäviä tehdessä tarvitsevat
apua. Eri ihmisillä kestää eri aika sisäistää luennon asia, joten
kertausmahdollisuus on eduksi.

On tärkeää, että videoissa on sisällysluettelo, jolloin opiskelijat löytävät
helposti tarvitseman kohdan videosta. Mitä helpommin videoilta saa halutun
tiedon, sitä vähemmän kirjallisesta materiaalista on tarve etsiä tiettyä asiaa.

Liian käsikirjoitetut luennot ovat huonoja, sillä silloin tilanne ei vastaa
todellisuutta ja opiskelijat voivat tylsistyä. Sopiva määrä improvisointia ja
mahdolliset virheet ovat hyväksi, koska silloin nähdään, että ohjelmoidessa on
luontaista tehdä välillä virheitä. Myös kääntäjän antamia virheilmoituksia on
syytä käsitellä, jotta ymmärretään, mistä virheet johtuvat.

Luennoilla on tärkeä näyttää myös refaktorointia. Aluksi tuetettu ratkaisu tekee
asiansa, mutta se ei välttämättä ole puhtain mahdollinen. Refaktorointi on
yleinen osa ohjelmointia, joten se tulee tuoda siten myös esille ja näyttää
miten refaktoroidaan.
